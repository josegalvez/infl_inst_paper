\documentclass[11pt]{article}

\usepackage{amsmath}
\usepackage{graphicx}
\usepackage[margin=1.in]{geometry}

\linespread{1.3}

\begin{document}
\section{Outline for Paper on Transverse Instabilities During Inflation and NG}
\subsection{Overview}
This is an outline of what I'd like to cover in a first paper on the work we've been doing on nonGaussianity from instabilities during inflation. Specifically I'd like to focus on the situation we've been looking at where $\Delta V$ turns on and off an instability in the transverse direction. And calculate the NG response to varying the parameters of $\Delta V$. Using
\begin{equation}
\Delta V(\phi,\chi) = A\left((\chi-\chi_0)^2(\chi+\chi_0)^2 - \chi_0^4 \right)f(\phi-\phi_p)
\end{equation}
we can vary the distance between minima, overall depth of the minima, and the profile of how the instability is turned on/off as clocked by $\phi$ (or anything else derivatbly from this form of potential).

The qualitative cases to consider are:
\begin{itemize}
  \item $\Delta V$ turns on and off again, or turns on and stays on,
  \item Time to cross $\Delta V$ long or short compared to dynamical time scale,
  \item Fields do or don't get trapped by $\Delta V$,
  \item Kick out of inflation or not.
\end{itemize}

The main take-away points I'd like a reader to have are:
\begin{enumerate}
  \item We have an in-out between equilibrium states with a burst of NG created during the transition,
  \item Lattice simulations provide a numerical experiment from which the NG can be measured,
  \item We find the NG to be in the form of prominences which are spatially localized.
\end{enumerate}

I'd like not to let the scope of the paper creep too much. We can make this two papers if need be.

\subsection{Plots and Results to Show}

I think some key plots we can show to tell the story are:
\begin{enumerate}
  \item Surface plot of the potential,
  \item Coulour map of $\zeta$ on a lattice slice, 
  \item Connected components of the n-pt functions as a function of time (and response to variations in $\Delta V$),
  \item Number count of positive/negative peaks as a function of peak height compared to the Gaussian field result (and response to variations in $Delta V$).
\end{enumerate}

\subsection{Bringing the Reader Up to Speed}
I think it would be a good idea to bring the reader up to speed on lattice simulations and the nonlinear $\zeta$ definition. Probably stick this in the introduction section.

In particular for lattice simulations emphasising the ways in which they differ from the traditional Fourier-centric approach ie:
\begin{itemize}
  \item Simulate and measure on a particular realization, as opposed to calculating ensemble statistics.
  \item Calculation done in real space, relates to phase information in the Fourier space picture.
  \item Tracks the nonlinear interactions in a nonperturbative way.
\end{itemize}

For the nonlinear $\zeta$ definition
\begin{itemize}
  \item The perturbative $\zeta$ is insufficient for studying phase transitions as perturbing around a homogeneous background breaks when the trajectories bifurcate.
  \item Not sure what else should be said here, but probably some justification for calculating $\zeta$ on the FRW used in the lattice calculation.
\end{itemize}

\subsection{Next Steps}
Here are the most immediate things to be done in order to get this paper written:
\begin{enumerate}
  \item Review and revise outline
\end{enumerate}


\end{document}
